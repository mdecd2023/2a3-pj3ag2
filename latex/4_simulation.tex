\chapter{2a3-pj3ag2製作心得}
%\renewcommand{\baselinestretch}{10.0} %設定行距
\section{江芷柔心得}
\section{李凱新心得}
這次使用SolidWorks 畫了一個球場,還使用onshape 畫了旗子上面還有數字,再匯入coppliasim 。我感覺對於使用場景更得心應手了呢!
\section{王翔楷心得}
這次作業我擔任建立球員及球場感測器還有最後Latex報告撰寫的工作,設計跑車模樣的球員讓我的繪圖技術透過這次的經驗更加精進,球場感測器的建立已經透過前幾次的作業徹底摸清楚,建設上非常的流暢也省時,最後是Latex報告的編寫,每次寫報告都能更精進使用latex的技術,總之這門課使我獲益良多。
\section{吳勁毅心得}
這次pj3的分組因為有8個人所以比之前更輕鬆,畢竟要負責的部分已經更少了,但途中還是遇到一些問題,像是有時無法連到主機,導致無法對戰,還有一些程式小問題,但我們會盡力解決,讓這個作業做得更好
\section{李學淵心得}
這次pj3我學習到了 lua在coppeliasim的各種用法,並整合了更多功能在程式之中,縮短程式的長度來簡潔美化。
\section{林秉賢心得}
這次的八人組合,我覺得做的東西有難度,但是如果要八個人一起做我覺得可以很快完成,我要很謝謝我的組員們很凱瑞我,沒有他們我覺得這個東西沒辦法處理得很好。
\section{張育銓心得}
這次的pj3 作業很難,大部分都是靠同組的同學完成的,在連網路也發現到了一些問題,並解決了它,非常的艱難,也很感謝同組的同學。
\section{張昱棠心得}
這次的作業裡的機械式計分器挑戰對我的繪圖挑戰真的很大,來回改良了好幾版,匯入coppliasim模擬作動。對我來說都是很棒的跳戰,我學到了很多解決事情的方法。
\newpage